\documentclass[table]{beamer}

% Use metropolis theme
\usepackage[progressbar=foot]{theme/beamerthememetropolis}
\usepackage{amsmath}
\usepackage{marvosym}
\usepackage{xcolor}
\usepackage{minted}
\usepackage{pdfpages}
\usepackage[nodayofweek,level]{datetime}
\usepackage{hyperref}

\title{When is Chinese New Year?}

\date{October 23, 2019}
\author{Joe Jevnik}
\institute{Boston Python}

\begin{document}
\maketitle

\section{Calendar Basics}

\begin{frame}{New Moon}
  \begin{itemize}
  \item<+-> \textit{ecliptic}: apparent path that the Sun takes as it
    revolves around the Earth
  \item<+-> \textit{new moon}: when the moon is completely dark
  \item<+-> \textit{new moon}: when the \textit{ecliptic longitude} of the Moon is equal to
    the \textit{ecliptic longitude} of the Sun
  \item<+-> when the Moon is at the same angle in the sky as the Sun but
    different heights
  \item<+-> if height is also equal, then it is an eclipse, thus the name
    \textit{ecliptic}
  \item<+-> \textit{lunation} or \textit{synodic month}: time between new moon
    and the next new moon.
  \item<+-> not totally regular due to many factors
  \item<+-> \textit{mean lunation} = 29.531 days
  \end{itemize}
\end{frame}

\begin{frame}{Year}
  \begin{itemize}
  \item<+-> \textit{tropical year}: winter solstice to winter solstice
  \item<+-> \textit{tropical year}: the amount of time it takes for the Sun to
    return to the same position in the sky
  \item<+-> tropical year = 365.242374 days
  \item<+-> 12 mean lunar months = 354.36707 days
  \item<+-> about 11 days short
  \end{itemize}
\end{frame}

\begin{frame}{Year (cont.)}
  If we define a year in terms of the...

  \begin{columns}[T]
    \column{0.35\textwidth}
    \begin{block}{Moon}
      \begin{itemize}
      \item easy to observe
      \item easy to measure progress
      \end{itemize}
    \end{block}

    \column{0.5\textwidth}
    \begin{block}{Sun}
      \begin{itemize}
      \item regular seasons
      \item easier to manage agricultural stuff
      \end{itemize}
    \end{block}
  \end{columns}
\end{frame}

\begin{frame}{Lunisolar Calendar}
  \begin{itemize}
  \item<+-> define a month as new moon to new moon
  \item<+-> add some correction factor to the months to stay aligned to the seasons
  \item<+-> similar to Hebrew calendar
  \end{itemize}
\end{frame}

\section{Chinese Calendar}

\begin{frame}{Solar Terms}
  \begin{itemize}
  \item<+-> 24 divisions of the Sun's ecliptic longitude into 24 equal parts
  \item<+-> even periods called \textit{major solar terms}
  \item<+-> odd periods called \textit{minor solar terms}
  \item<+-> both equinoxes and both solstices occur on major solar terms
  \end{itemize}
\end{frame}

\begin{frame}{Rule 1}
  \begin{itemize}
  \item<+-> calculations are based on 120° East
  \item<+-> determines solar noon
  \item<+-> tzinfo Asia/Shanghai
  \item<+-> changed from 116°25' East in 1928, which is about a 14 minute shift
  \item<+-> need to account for this when computing holiday dates in the past
  \end{itemize}
\end{frame}

\begin{frame}{Rule 2}
  \begin{itemize}
  \item<+-> days start at midnight
  \item<+-> days are 24 hours long
  \end{itemize}
\end{frame}

\begin{frame}{Rule 3}
  \begin{itemize}
  \item<+-> the day on which a new moon occurs is the first of the new month
  \item<+-> if the conjunction is at 23:59:59, that day is still the first of the
    month
  \item<+-> months are either 30 or 29 days (except for intercalary months)
  \item<+-> the same month can be different lengths in different years
  \item<+-> uses computations of  time instead of observations as of
    around 621 BCE
  \end{itemize}
\end{frame}

\begin{frame}{Chinese Year}
  \begin{itemize}
  \item<+-> two kinds of years: \textit{solar year} and \textit{lunar year}
  \item<+-> \textit{solar year}: winter solstice to winter solstice
  \item<+-> contains 12 lunar months and about 11 days
  \item<+-> \textit{lunar year}: Chinese new year to Chinese new year
  \item<+-> contains either 12 or 13 lunar months
  \item<+-> the extra lunar month is an intercalary month to keep the seasons
    aligned
  \end{itemize}
\end{frame}

\begin{frame}{Rule 4}
  \begin{itemize}
  \item<+-> the winter solstice falls in month 11
  \item<+-> a solar year is a leap solar year if there are 12 complete lunar
    months between the two 11th months at the beginning and end of the solar
    year
  \end{itemize}
\end{frame}

\begin{frame}{Rule 5}
  \begin{itemize}
  \item<+-> in a leap solar year, the first month that does not contain a minor
    solar term is the leap month
  \item<+-> the leap month extends the duration of the previous month
  \item<+-> any month can be a leap month
  \item<+-> this rule only applies to leap solar years, so rule 4 takes precedence
  \item<+-> this leads to the Y2033 issue
  \item<+-> possible to have a month with no minor solar term that is
    \textit{not} a leap month
  \item<+-> called a fake leap month
  \end{itemize}
\end{frame}

\section{Code}

\begin{frame}{NOVAS}
  \begin{itemize}
  \item<+-> Naval Observatory Vector Astrometry Software
  \item<+-> provides calculations for the positions of local objects in
    different coordinate systems
  \item<+-> paid for by taxes, cannot be copyrighted, it is simply free
  \item<+-> requires an \textit{ephemeris}
  \item<+-> can use the JPL ephemerides files
  \end{itemize}
\end{frame}

\begin{frame}{Terrestrial Time and Julian Date}
  \begin{itemize}
  \item<+-> \textit{Julian Day} is exactly 86400 atomic
    seconds
  \item<+-> drifts from actual days partly because the earth is slowing down
  \item<+-> \textit{Julian Date} is the number of julian days since the
    \textit{Julian Period} as a real number
  \item<+-> \textit{Julian Days} start at noon
  \item<+-> \textit{Terrestrial Time} is an ideal measure of time that doesn't
    account for irregularities in the rotation of the earth.
  \item<+-> \textit{Julian Date TT} is the number of julian days since the
    Julian Period
  \end{itemize}
\end{frame}

\begin{frame}[fragile]{UTC to JD TT Corrections}
  \begin{minted}{python}
# table from US Naval Observatory
utc_to_jd_tt.corrections = np.array(
    [
        (T(1700, 1, 1), 32.184),
        (T(1973, 2, 1), 43.4724),
        (T(1973, 3, 1), 43.5648),
        # ...
        (T(2025, 9, 1), 72.0),
        (T(2026, 1, 1), 72.0),
    ],
    dtype=[('time', 'M8[D]'), ('julian_days', 'f8')],
)
# seconds in a julian day
utc_to_jd_tt.corrections['julian_days'] /= 60 * 60 * 24
  \end{minted}
\end{frame}

\begin{frame}[fragile]{UTC to JD TT}
  \begin{minted}{python}
def utc_to_jd_tt(ts):
    corrections = utc_to_jd_tt.corrections
    jd_utc = (
       ts.astype('M8[s]').view('i8') / 86400 + 2440587.5
    )
    correction_ix = np.searchsorted(
        corrections['time'],
        ts,
        side='right',
    )
    correction_ix[correction_ix == len(corrections)] -= 1
    return (
       jd_utc + corrections['julian_days'][correction_ix]
    )
  \end{minted}
\end{frame}

\begin{frame}[fragile]{Getting New Moons}
  \begin{minted}{python}
def compute_ecliptic_longitude(ob, minutes):
    longitude_degrees = np.array([
        novas.equ2ecl(
            jd_tt,
            # right ascension and declination
            *novas.app_planet(jd_tt, ob)[:2],
            # true equator and equinox of date
            coord_sys=1,
        )[0]
        for jd_tt in utc_to_jd_tt(minutes)
    ])
    return np.deg2rad(longitude_degrees)
  \end{minted}
\end{frame}

\begin{frame}[fragile]{Getting New Moons (Cont.)}
  \begin{minted}{python}
moon = novas.make_object(0, 11, 'Moon', None)
lunar_el = compute_ecliptic_longitude(moon, minutes)

sun = novas.make_object(0, 10, 'Sun', None)
solar_el = compute_ecliptic_longitude(sun, minutes)
  \end{minted}
\end{frame}

\begin{frame}{When is Chinese New Year?}
  \begin{itemize}
  \item[] \formatdate{25}{01}{2020}
  \item[] \formatdate{12}{02}{2021}
  \item[] \formatdate{01}{02}{2022}
  \item[] \formatdate{22}{01}{2023}
  \item[] \formatdate{10}{02}{2024}
  \item[] \formatdate{29}{01}{2025}
  \item[] \formatdate{17}{02}{2026}
  \item[] \formatdate{06}{02}{2027}
  \item[] \formatdate{26}{01}{2028}
  \item[] \formatdate{13}{02}{2029}
  \item[] \formatdate{03}{02}{2030}
  \end{itemize}
\end{frame}

\begin{frame}{Thanks}
  \begin{block}{Implementation}
    \url{https://github.com/quantopian/trading_calendars/blob/master/etc/lunisolar}
  \end{block}
  \begin{block}{References}
    \begin{itemize}
    \item Aslaksen, Helmer. "The Mathematics of the Chinese Calendar"
    \item NOVAS
    \item Jean H. Meeus. 1991. Astronomical Algorithms. Willmann-Bell,
      Incorporated.
    \end{itemize}
  \end{block}
\end{frame}

\end{document}